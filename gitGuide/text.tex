%!TEX TS-program = xelatex
%!TEX encoding = UTF-8 Unicode

\documentclass{article}
\usepackage{listings,framed,xcolor,xeCJK,graphicx,float,amsmath}
\usepackage[
			colorlinks,
			linkcolor=black,
			pdfborder={0 0 0},
			CJKbookmarks=true
			]{hyperref}

\setmainfont[BoldFont=SimHei]{SimSun}
\definecolor{lgray}{rgb}{0.95,0.95,0.95}
\lstset{
		basicstyle=\tt,
%		numbers=left,
%		stepnumber=1,
		frame=L,
		columns=fullflexible,
		breaklines=true,
		keywordstyle=\color{blue},
		backgroundcolor=\color{lgray},
		stringstyle=\color{red}
		boxpos=t
		}

\title{A Guide to Git and Github}
\author{Internet Service Department}
\date{\today}
\begin{document}
	\maketitle
	\tableofcontents
	\newpage
	\section{安装与配置} % (fold)
	\label{sec:安装与配置}
		\begin{enumerate}
			\item 首先,注册一个\href{http://github.com/}{Github}帐号。
			\\在\href{http://msysgit.github.io/}{此处}下载并安装Git for Windows,通常可以直接使用默认配置。安装好之后可以在「开始」菜单中找到「Git Bash」和「Git GUI」两个程序。
			\item 打开「Git Bash」,首先设置用户名和邮箱,在命令行中键入如下命令:
			\begin{quote}
				\begin{lstlisting}
$ git config --global user.name "Your Name"
$ git config --global user.email yourmail@server.com
				\end{lstlisting}
			\end{quote}
			\item 然后创建SSH密钥
			\begin{quote}
				\begin{lstlisting}
$ ssh-keygen -C 'yourmail@server.com' -t rsa
				\end{lstlisting}
			\end{quote}
			\item Git Bash会询问储存密钥的路径,方便起见可以使用其默认路径,在windows下即为用户文件夹。生成成功后可以在之前设定的路径中找到{\tt id\_rsa.pub}文件,用记事本打开,文件中的字符即为所需的SSH密钥。
			\item 登录Github首页,点击Account Setting $\to$ SSH keys $\to$ Add SSH key。title可以随便填写,以可以判断这是自己电脑为原则。在key的输入框中粘贴{\tt id\_rsa.pub}文件中的内容,点击Apply即可。
			\\测试与 GitHub 是否连接成功:
			\begin{quote}
				\begin{lstlisting}
$ SSH -v git@github.com
				\end{lstlisting}
			\end{quote}
			返回
			\begin{quote}
				\begin{lstlisting}
$ Are you sure you want to continue connecting (yes/no)?
				\end{lstlisting}
			\end{quote}
			时,输入yes。
			最后,返回
			\begin{quote}
				\begin{lstlisting}
You've successfully authenticated, but GitHub does not provide shell access.
				\end{lstlisting}
			\end{quote}
			时,SSH密钥配置成功。
		\end{enumerate}
	% section 安装与配置 (end)
	\section{Git基本概念} % (fold)
	\label{sec:git基本概念}
		\subsection{仓库(Repositories)}
			仓库即为储存一个项目的代码的目录。仓库里包括了一个项目的全部代码和README文件。可以用本地现有的代码新建一个仓库,并上传至Git,也可以将Git上已有的仓库克隆(clone)到本地,进行操作。
			\par从本地新建一个仓库的操作是:
				\begin{quote}
					\begin{lstlisting}
$ cd \c\your\repo\path
$ git init name_of_repo
					\end{lstlisting}
				\end{quote}
			\par 接着,用{\tt git add}命令来添加改项目中需要追踪的文件(如一个TeX项目中,需要追踪的文件即为.tex文件,.pdf文件和其他的素材文件,而一些临时文件则不需要进行追踪)。如果需要永久性的忽略某一类型的文件,可以打开目录中的{\tt .gitignore}文件,并将不需要跟踪的文件添加进去。
			\begin{quote}
				\begin{lstlisting}
$ git add *.* $ git add *.* $ git add *.* ......
				\end{lstlisting}
			\end{quote}
			也可以直接添加全部文件(在{\tt .gitignore}中添加过的文件会被忽略)
			\begin{quote}
				\begin{lstlisting}
$ git add .
				\end{lstlisting}
			\end{quote}
			\par 从现有的仓库克隆的方法为:
			\begin{quote}
				\begin{lstlisting}
$ git clone git://github.com/....../*.git
				\end{lstlisting}
			\end{quote}
			\par 现有的仓库的克隆地址可以在项目的Github页面上找到。这会在当前目录下创建一个名为“*”的目录,其中包含一个 .git 的目录文件,用于保存下载下来的所有版本记录,然后从中取出最新版本的文件拷贝。在仓库的目录下键入{\tt git status}命令,可以查看已跟踪和未跟踪的文件列表。
			\par{\bf\textcolor{red}{有多个项目时,在操作前必须用{\tt cd}命令进入需要操作的项目的目录。}}
		\subsection{提交与推送(commit and push)} % (fold)
		\label{sub:提交与推送_commit_and_push_}
			\par 在git的体系中,本地仓库由三部分组成:工作目录(Working Directory),即实际存在的文件夹。缓存区(Index),用来临时储存项目的改动。和HEAD,即最后一次正式提交的结果。{\tt git add}命令的意义就是将有效改动添加至缓存区,准备提交。
			\par 当在本地修改项目之后,我们需要提交和推送这些修改。
			\par 首先是用{\tt git add}命令来添加需要提交的文件,然后键入
			\begin{quote}
				\begin{lstlisting}
$ git commit -m `commit message'
				\end{lstlisting}
			\end{quote}
			提交时需要为这次提交写一些简短的说明(commit message),说明这次修改了哪些地方。{\bf 每次提交相当于为项目创建了一个新的版本}。提交后,所有的改动都提交至了HEAD。
			\par 如果想省去{\tt git add}这一步,可以直接用
			\begin{quote}
				\begin{lstlisting}
git commit -a -m 'commit message'
				\end{lstlisting}
			\end{quote}
			{\tt commit -a}相当于:
			\begin{itemize}
				\item 自动地add所有改动的代码,使得所有的开发代码都列于Index中
				\item 自动地删除那些在Index中但在工作目录中已经被删除的文件
				\item 执行commit命令来提交
			\end{itemize}
			\textcolor{red}{$\bf\clubsuit$}~但是此命令并不会提交新添加的文件,所以如果有增加文件请乖乖使用{\tt git add}。
			\par 如果是本地的新建仓库,从来没有推送到服务器过,则应该使用如下命令
			\begin{quote}
				\begin{lstlisting}
$ git remote add origin git@github.com:....../*.git
$ git push origin master
				\end{lstlisting}
			\end{quote}
			\par 结束工作后,键入以下的命令将本地的改动提交到服务器
			\begin{quote}
				\begin{lstlisting}
$ git push origin master
				\end{lstlisting}
			\end{quote}
			此处的origin为服务器的别名,origin是默认值,也可以在remote的时候自行修改,如果在一个项目remote了多个服务器(如Github和Bitbucket),该命令可以控制push到哪一个服务器。mster为master分支,关于分支会在下面一节中讲到。
		% subsection 提交与推送_commit_and_push_ (end)
		\subsection{更新与恢复} % (fold)
		\label{sub:更新与恢复}
			有时在本地仓库的代码旧于服务器端的代码时(这在多人开发或者在多台设备上进行开发时比较常见),可以使用如下命令
			\begin{quote}
				\begin{lstlisting}
$ git pull
				\end{lstlisting}
			\end{quote}
			该命令会抓取服务器端的最新版本的代码并将本地的代码更新。
			\par 在git的体系中,所有提交过的更改都会被保存下来,并且都是可恢复的,即在任何时刻都可以将代码回滚到过去的任何一个版本
		% subsection 更新与恢复 (end)
	% section git基本概念 (end)
	\section{协作与远端操作} % (fold)
	\label{sec:协作与远端操作}
		\subsection{分支(branches)} % (fold)
		\label{sub:分支}
			使用
			\begin{quote}
				\begin{lstlisting}
$ git branch branch_name
				\end{lstlisting}
			\end{quote}
			命令创建一个分支,分支可以理解为一些相对独立的开发路线。在git中,分支可以随时无限量的被创建和被合并。举个例子,假设在新官网开发项目中,A先生和B先生分别从作为开发主线的master分支中fork(关于fork,会在下一小节中讲到)了一个分支(我们可以称之为a分支和b分支),分别用来写前台代码和后台代码,此时a分支和b分支的开发进程是相对独立的,在A先生和B先生完成了开发之后,他们向服务器push了自己的分支,这样别人就可以抓取这些代码了。然后,可以使用
			\begin{quote}
				\begin{lstlisting}
$ git checketout branch_name
				\end{lstlisting}
			\end{quote}
			命令来切换分支。然后,使用
			\begin{quote}
				\begin{lstlisting}
$ git merge another_branch_name
				\end{lstlisting}
			\end{quote}
			可以合并分支,比如A先生键入了
			\begin{quote}
				\begin{lstlisting}
$ git checketout b 
$ git merge a
$ git branch -d a
				\end{lstlisting}
			\end{quote}
			这些命令的含义为:进入b分支,将a分支合并入b分支,删除a分支。
			\par 如果要删除的分支没有被合并到其它分支中去,那么就不能用{\tt git branch -d}来删除它,需要改用{\tt git branch -D}来强制删除。
			\par 在分支合并时,有时会出现合并冲突,即两个分支分别修改了同一个文件的同一个地方,此时git会要求手动解决冲突。
			\par 打开产生冲突的文件,可以看到如下形式的冲突解决标记:
			\begin{quote}
				\begin{lstlisting}
<<<<<<< a:example.html

 AAAAAAA

 ======= 

 BBBBBBB

>>>>>>> b:example.html
				\end{lstlisting}
			\end{quote}
			分割线的上面和下面为冲突行在a分支和欲合并的b分支中的内容,手动删除要抛弃的内容之后,运行一次{\tt git add}来标记问题已解决。然后可以运行一次{\tt git status}来确认一下。解决冲突问题之后,就可以提交commit了。

			~\\\noindent\textbf{参考阅读}:\url{http://www.open-open.com/lib/view/open1328069889514.html}
		% subsection 分支 (end)
		\subsection{Fork 与克隆(clone)} % (fold)
		\label{sub:fork_与克隆_clone_}
			git中最重要的概念之一就是fork,fork的意义为,将别人的仓库作为一个分支拷贝到自己的仓库,但是和自己创建的分支不同的是,fork过来的项目是作为一个独立的项目存在于你自己的账户下的,并不能随意的与原项目合并(但并不是不能合并)。
			\par fork的方法为,在想要fork的项目页面点击fork按钮。fork成功之后,使用
			\begin{quote}
				\begin{lstlisting}
$ git clone git@github.com:.../*/git
				\end{lstlisting}
			\end{quote}
			命令将项目克隆到本地,就可以开始进行开发了。
			申请与原项目合并的方法为,在项目页面使用pull request功能。项目所有者通过后即可合并

			~\\注:这篇教程的内容也作为一个项目储存在github上,项目主页为\url{https://github.com/lamons/guides},可以拿来做fork和其他一些功能的实验。
		% subsection fork_与克隆_clone_ (end)
	% section 协作与远端操作 (end)
	\section{检阅代码} % (fold)
	\label{sec:检阅代码}
		\begin{quote}
			\begin{lstlisting}
$ git diff --cached
$ git diff branch_name
$ git diff HEAD HEAD^
			\end{lstlisting}
		\end{quote}
		\par{\tt git diff}是用于查看修改的代码信息,上面的代码中,{\tt git diff --cached}用来查看缓存区中的文件相对于HEAD的差异,{\tt git diff branch\_name}用于查看某一分支相对于master分支的差异,{\tt git diff HEAD HEAD$\hat{}$}用于查看当前的HEAD与前一版本的HEAD之间的差异(一个$\hat{}$代表了向前推一个版本)。
		\begin{quote}
			\begin{lstlisting}
$ git log
$ git log -p
			\end{lstlisting}
		\end{quote}
		\par{\tt git log}命令用于显示开发日志,可以显示出项目从早到晚所有commit的信息。{\tt git log -p}则出了commit信息,还可以调出所有的代码的修改历史。
		\begin{quote}
			\begin{lstlisting}
$ git tag v1.0
$ git tag -l
			\end{lstlisting}
		\end{quote}
		{\tt git tag}命令可以用来为项目标记版本号。{\tt git tag -l}可以查看最近的版本号。
	% section 检阅代码 (end)
	\section{后悔药} % (fold)
	\label{sec:后悔药}
		git的一个好处就是,它会保存你所有的完整的历史版本,所以在代码出现问题的时候,随时可以将代码回滚到任意一个历史版本。
		\begin{quote}
			\begin{lstlisting}
$ git checkout -- <filename>
			\end{lstlisting}
		\end{quote}
		这条命令可以用HEAD中的文件替替换掉你的工作目录中的文件,已添加到缓存区的改动,以及新文件,都不受影响。
		\par 假如你想要丢弃你所有的本地改动与提交,可以到服务器上获取最新的版本并将你本地主分支指向到它:
		\begin{quote}
			\begin{lstlisting}
$ git fetch origin
$ git reset --hard origin/master
			\end{lstlisting}
		\end{quote}
		可以用以下的代码来回滚到过去的版本
		\begin{quote}
			\begin{lstlisting}
$ git revert HEAD
$ git revert HEAD^
$ git revert commit <commit code>
			\end{lstlisting}
		\end{quote}
		{\tt git revert commit <commit code>}命令用来回滚至指定的版本,commit code可以在github项目页面上找到,类似bd22a27dbab52c119c7ff66a1c79a8491f82e5d6这样。
	% section 后悔药 (end)
\end{document}